\documentclass[a4paper,11pt]{jsarticle}


% 数式
\usepackage{amsmath,amsfonts}
\usepackage{braket}
\usepackage{bm}
% 画像
\usepackage[dvipdfmx]{graphicx}

\usepackage{url}


\begin{document}

\title{情報セキュリティ学特論レポート

3者間DH鍵共有}
\author{園田継一郎}
\date{\today}
\maketitle

\section{はじめに}
2者間で鍵を共有する手法として, DH鍵共有がある.
DH鍵共有では素数$p$ (以下は全て${\rm mod}\hspace{3pt} p$とする)
と生成元$g$を決め, 
Aさん, Bさん, Cさんがそれぞれが秘密の整数値
$a, b, c$を持っている.
$g^a, g^b, g^c$は公開されるので, 
Aさん, Bさんの2者間であれば$g^{ab}$を共有できる.
3者間で同じ値を共有したいとき,
例えばAさんは$(g^b \cdot g^c)^a = g^{ab + ac}$
を計算できるが, BさんとCさんは$g^{ab + ac}$
を計算できない.
共有できそうな値として, $g^{a + b + c}, g^{abc}$が挙げられる.
このうち$g^{a + b + c}$は$g^a \cdot g^b \cdot g^c$で誰でも計算
できてしまうので秘密鍵として使えない.
$g^{abc}$を共有できることが理想だが, 
それぞれが知っている情報で$g^{abc}$は計算できない.
以下では, 3者間で鍵を共有するための手法を紹介する.

\section{3者間DH鍵共有}
3者間DH鍵共有には, 楕円曲線上のペアリングという演算が使われる.
ペアリングは, 楕円曲線$E$上の2個の点の組から
ある有限体$\mathbb{F}_p$への写像である\cite{bib1}.
$P, Q$を$E$上の点, $g$を生成元とすると, ペアリング$e$
は以下のように定義される.
\[
  \begin{array}{rccc}
    e \colon & E \times E & \longrightarrow & \mathbb{F}_p \\
            & \rotatebox{90}{$\in$} & & \rotatebox{90}{$\in$} \\
            & (P, Q) & \longmapsto & g^{S(P, Q)}
  \end{array}
\]
ここで$S(P, Q)$とは, 位置ベクトル$P, Q$で張られる
平行四辺形の面積である. ただし, $Q$が$P$の半時計回りに位置する場合は
正となり, そうでなければ負となる. 辺の長さを$a$倍したとき, 
面積も$a$倍されるので, $a, b \in \mathbb{Z}$としたとき, 
$S$について以下が成り立つ. 
\[
  S(aP, bQ) = abS(P, Q)
\]
つまり, ペアリングでは
\[
  e(aP, bQ) = g^{S(aP, bQ)} = g^{abS(P, Q)} =
  \left(g^{S(P, Q)}\right)^{ab} = e(P, Q)^{ab}
\]
が成り立つ. この性質を使えば, 以下の方法で3者間鍵共有ができる.
\begin{enumerate}
  \item 楕円曲線上の$P, Q$を固定してAさん, Bさん, Cさんで共有する.
  \item それぞれ秘密の整数値$a, b, c$を持ち, $(aP, aQ), (bP, bQ), (cP, cQ)$を公開する.
  \item Aさんは$e(bP, cQ)^a = e(P, Q)^{abc}$を計算する.
    Bさん, Cさんも同様に$e(P, Q)^{abc}$を計算する.
\end{enumerate}

\section{まとめ}
ペアリングを用いることで, 3者以上との鍵共有ができ,
マルチキャストしやすくなる.
しかし, まだ実用的ではない.

\begin{thebibliography}{99}
  \bibitem{bib1}光成 滋生「クラウドを支えるこれからの暗号技術」秀和システム(2015) \url{https://github.com/herumi/ango/raw/master/ango.pdf}
\end{thebibliography}

\end{document}
