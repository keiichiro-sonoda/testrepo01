\documentclass[a4paper,11pt]{jsarticle}


% 数式
\usepackage{amsmath,amsfonts}
\usepackage{braket}
\usepackage{bm}
% 画像
\usepackage[dvipdfmx]{graphicx}

\usepackage{url}


\begin{document}

\title{情報セキュリティ学特論レポート

3者間DH鍵共有}
\author{園田継一郎}
\date{\today}
\maketitle

\section{はじめに}
DH鍵共有では, 2者間でしか鍵の共有ができない.
DH鍵共有は, 以下のように行う.
複数のメッセージに同じメッセージを送る場合, 
3者間で鍵共有ができれば便利である.

\section{3者間DH鍵共有}
3者間DH鍵共有には, 楕円曲線上のペアリングという演算が使われる.
ペアリングは, 楕円曲線$E$上の2個の点の組から
ある有限体$F_q$への写像である\cite{bib1}.
$P, Q \in E$, $g$を生成元とすると, ペアリング$e$
は以下のように定義される.
\[
  \begin{array}{rccc}
    e \colon & E & \longrightarrow & E \\
            & \rotatebox{90}{$\in$} & & \rotatebox{90}{$\in$} \\
            & (P, Q) & \longmapsto & g^{S(P, Q)}
  \end{array}
\]
ここで$S(P, Q)$とは, 楕円曲線$E$上の位置ベクトル$P, Q$で張られる
平行四辺形の面積である. ただし, $Q$が$P$の半時計回りに位置する場合は
正となり, そうでなければ負となる. $P$を$a (a\in \mathbb{Z})$倍したとき
面積も$a$倍されるため, $S$は第一成分について線形性を持つ.
第二成分についても同様である. $a, b \in \mathbb{Z}$として式で表すと, 
\[
  S(aP, bQ) = abS(P, Q)
\]
となる. 

\section{まとめ}
ペアリングを用いることで, 3者以上との鍵共有ができ,
マルチキャストしやすくなる.
しかし, まだ実用的ではない.

\begin{thebibliography}{99}
  \bibitem{bib1}光成 滋生「クラウドを支えるこれからの暗号技術」秀和システム(2015) \url{https://github.com/herumi/ango/raw/master/ango.pdf}
\end{thebibliography}

\end{document}
