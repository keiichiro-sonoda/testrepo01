\documentclass[a4paper,11pt]{jsarticle}


% 数式
\usepackage{amsmath,amsfonts}
\usepackage{braket}
\usepackage{bm}
% 画像
\usepackage[dvipdfmx]{graphicx}


\begin{document}

\title{応用関数解析特論レポート}
\author{園田継一郎}
\date{\today}
\maketitle

\section{}
$\mathbb{C}^2$の標準基底
\[
  e_1 = \begin{bmatrix}
    1 \\ 0
  \end{bmatrix}, 
  e_2 = \begin{bmatrix}
    0 \\ 1
  \end{bmatrix}
\]
について${\rm H}e_1, {\rm H}e_2$を計算すると
\begin{eqnarray*}
  {\rm H}e_1 = \frac{1}{\sqrt{2}} \begin{bmatrix}
    1 & 1 \\ 1 & -1
  \end{bmatrix} \begin{bmatrix}
    1 \\ 0
  \end{bmatrix} = \frac{1}{\sqrt{2}} \begin{bmatrix}
    1 \\ 1
  \end{bmatrix} \\
  {\rm H}e_2 = \frac{1}{\sqrt{2}} \begin{bmatrix}
    1 & 1 \\ 1 & -1
  \end{bmatrix} \begin{bmatrix}
    0 \\ 1
  \end{bmatrix} = \frac{1}{\sqrt{2}} \begin{bmatrix}
    1 \\ -1
  \end{bmatrix}
\end{eqnarray*}
となる. それぞれの内積は
\begin{eqnarray*}
  \braket{{\rm H}e_1, {\rm H}e_1}
  &=& \frac{1}{2}(1 + 1) = 1 = \braket{e_1, e_1} \\
  \braket{{\rm H}e_1, {\rm H}e_2}
  &=& \frac{1}{2}(1 - 1) = 0 = \braket{e_1, e_2} \\
  \braket{{\rm H}e_2, {\rm H}e_1}
  &=& \frac{1}{2}(1 - 1) = 0 = \braket{e_2, e_1} \\
  \braket{{\rm H}e_2, {\rm H}e_2}
  &=& \frac{1}{2}(1 + 1) = 1 = \braket{e_2, e_2} \\
\end{eqnarray*}
となり, 標準基底について${\rm H}$はユニタリ作用素の条件を満たす.
${}^\forall x, y \in \mathbb{C}^2$は$e_1, e_2$
の線形結合で表せるので, 
\[
  \braket{{\rm H}x, {\rm H}y} = \braket{x, y}
\]
が得られる. よって${\rm H}$はユニタリ作用素である.

\section{}
$\mathbb{C}^3$の正規直交基底の1つとして, 
\[
  e_1 = \frac{1}{\sqrt{14}}\begin{bmatrix}
    1 \\ 2 \\ 3
  \end{bmatrix}, 
  e_2 = \frac{1}{\sqrt{182}}\begin{bmatrix}
    13 \\ -2 \\ -3
  \end{bmatrix},
  e_3 = \frac{1}{\sqrt{13}}\begin{bmatrix}
    0 \\ 3 \\ -2
  \end{bmatrix}
\]
が挙げられる.

\section{}

状態と正規直行基底の内積を計算すると
\begin{eqnarray*}
  |\braket{\xi_1, \psi}|^2
  &=& \left(\frac{1}{6}(-2 + 2 + 1 + 0)\right)^2
  = \left(\frac{1}{6}\right)^2 = \frac{1}{36} \\
  |\braket{\xi_2, \psi}|^2
  &=& \left(\frac{1}{6}(2 - 2 + 1 + 0)\right)^2
  = \left(\frac{1}{6}\right)^2 = \frac{1}{36} \\
  |\braket{\xi_3, \psi}|^2
  &=& \left(\frac{1}{6}(2 + 2 - 1 + 0)\right)^2
  = \left(\frac{1}{2}\right)^2 = \frac{1}{4} \\
  |\braket{\xi_4, \psi}|^2
  &=& \left(\frac{1}{6}(2 + 2 + 1 - 0)\right)^2
  = \left(\frac{5}{6}\right)^2 = \frac{25}{36}
\end{eqnarray*}
となる. 公理2より, 数値1, 2, 3, 4が検出される確率はそれぞれ
$\frac{1}{36}, \frac{1}{36}, \frac{1}{4}, \frac{25}{36}$
である.

\section{}
$m, n \in \mathbb{N}$とし, 集合の元どうしで内積を計算する.
$x \in [0, 2\pi]$のとき$\cos{nx}, \sin{nx} \in [0, 1]$なので, 
\begin{eqnarray*}
  \overline{\cos{nx}} = \cos{nx} \\
  \overline{\sin{nx}} = \sin{nx} \\
\end{eqnarray*}
が成立する. まず, 元が同じ場合の内積を計算する.
\begin{eqnarray*}
  &&\braket{\frac{1}{\sqrt{\pi}}\cos{nx}, \frac{1}{\sqrt{\pi}}\cos{nx}} \\
  &=& \int_{0}^{2\pi}{\overline{\frac{1}{\sqrt{\pi}}\cos{nx}}\frac{1}{\sqrt{\pi}}\cos{nx}}dx \\
  &=& \frac{1}{\pi}\int_{0}^{2\pi}{\cos^{2}{nx}}dx
\end{eqnarray*}

\end{document}
