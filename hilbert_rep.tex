\documentclass[a4paper,11pt]{jsarticle}


% 数式
\usepackage{amsmath,amsfonts}
\usepackage{braket}
\usepackage{bm}
% 画像
\usepackage[dvipdfmx]{graphicx}


\begin{document}

\title{応用関数解析特論レポート}
\author{園田継一郎}
\date{\today}
\maketitle

\section{}
$\mathbb{C}^2$の標準基底
\[
  e_1 = \begin{bmatrix}
    1 \\ 0
  \end{bmatrix}, 
  e_2 = \begin{bmatrix}
    0 \\ 1
  \end{bmatrix}
\]
で$\{{\rm H}e_1, {\rm H}e_2\}$がCONSになることを示す.
${\rm H}e_1, {\rm H}e_2$を計算すると
\begin{eqnarray*}
  {\rm H}e_1 = \frac{1}{\sqrt{2}} \begin{bmatrix}
    1 & 1 \\ 1 & -1
  \end{bmatrix} \begin{bmatrix}
    1 \\ 0
  \end{bmatrix} = \frac{1}{\sqrt{2}} \begin{bmatrix}
    1 \\ 1
  \end{bmatrix} \\
  {\rm H}e_2 = \frac{1}{\sqrt{2}} \begin{bmatrix}
    1 & 1 \\ 1 & -1
  \end{bmatrix} \begin{bmatrix}
    0 \\ 1
  \end{bmatrix} = \frac{1}{\sqrt{2}} \begin{bmatrix}
    1 \\ -1
  \end{bmatrix}
\end{eqnarray*}
となるので, 内積は
\begin{eqnarray*}
  \braket{{\rm H}e_1, {\rm H}e_1}
  = \frac{1}{2}(1 + 1) = 1 \\
  \braket{{\rm H}e_1, {\rm H}e_2}
  = \frac{1}{2}(1 - 1) = 0 \\
  \braket{{\rm H}e_2, {\rm H}e_1}
  = \frac{1}{2}(1 - 1) = 0 \\
  \braket{{\rm H}e_2, {\rm H}e_2}
  = \frac{1}{2}(1 + 1) = 1 \\
\end{eqnarray*}

\end{document}
