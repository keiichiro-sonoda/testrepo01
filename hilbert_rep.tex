\documentclass[a4paper,11pt]{jsarticle}


% 数式
\usepackage{amsmath,amsfonts}
\usepackage{braket}
\usepackage{bm}
% 画像
\usepackage[dvipdfmx]{graphicx}

\usepackage{physics}

\begin{document}

\title{応用関数解析特論レポート}
\author{園田継一郎}
\date{\today}
\maketitle

\section{}
$\mathbb{C}^2$の標準基底
\[
  e_1 = \begin{bmatrix}
    1 \\ 0
  \end{bmatrix}, 
  e_2 = \begin{bmatrix}
    0 \\ 1
  \end{bmatrix}
\]
について${\rm H}e_1, {\rm H}e_2$を計算すると
\begin{eqnarray*}
  {\rm H}e_1 = \frac{1}{\sqrt{2}} \begin{bmatrix}
    1 & 1 \\ 1 & -1
  \end{bmatrix} \begin{bmatrix}
    1 \\ 0
  \end{bmatrix} = \frac{1}{\sqrt{2}} \begin{bmatrix}
    1 \\ 1
  \end{bmatrix} \\
  {\rm H}e_2 = \frac{1}{\sqrt{2}} \begin{bmatrix}
    1 & 1 \\ 1 & -1
  \end{bmatrix} \begin{bmatrix}
    0 \\ 1
  \end{bmatrix} = \frac{1}{\sqrt{2}} \begin{bmatrix}
    1 \\ -1
  \end{bmatrix}
\end{eqnarray*}
となる. それぞれの内積は
\begin{eqnarray*}
  \left<{\rm H}e_1, {\rm H}e_1\right>
  &=& \frac{1}{2}(1 + 1) = 1 = \left<e_1, e_1\right> \\
  \left<{\rm H}e_1, {\rm H}e_2\right>
  &=& \frac{1}{2}(1 - 1) = 0 = \left<e_1, e_2\right> \\
  \left<{\rm H}e_2, {\rm H}e_1\right>
  &=& \frac{1}{2}(1 - 1) = 0 = \left<e_2, e_1\right> \\
  \left<{\rm H}e_2, {\rm H}e_2\right>
  &=& \frac{1}{2}(1 + 1) = 1 = \left<e_2, e_2\right> \\
\end{eqnarray*}
となり, 標準基底について${\rm H}$はユニタリ作用素の条件を満たす.
${}^\forall x, y \in \mathbb{C}^2$は$e_1, e_2$
の線形結合で表せるので, 
\[
  \left<{\rm H}x, {\rm H}y\right> = \left<x, y\right>
\]
が得られる. よって${\rm H}$はユニタリ作用素である.

\section{}
$\mathbb{C}^3$の正規直交基底の1つとして, 
\[
  e_1 = \frac{1}{\sqrt{14}}\begin{bmatrix}
    1 \\ 2 \\ 3
  \end{bmatrix}, 
  e_2 = \frac{1}{\sqrt{182}}\begin{bmatrix}
    13 \\ -2 \\ -3
  \end{bmatrix},
  e_3 = \frac{1}{\sqrt{13}}\begin{bmatrix}
    0 \\ 3 \\ -2
  \end{bmatrix}
\]
が挙げられる.

\section{}

状態と正規直行基底の内積を計算すると
\begin{eqnarray*}
  |\left<\xi_1, \psi\right>|^2
  &=& \left(\frac{1}{6}(-2 + 2 + 1 + 0)\right)^2
  = \left(\frac{1}{6}\right)^2 = \frac{1}{36} \\
  |\left<\xi_2, \psi\right>|^2
  &=& \left(\frac{1}{6}(2 - 2 + 1 + 0)\right)^2
  = \left(\frac{1}{6}\right)^2 = \frac{1}{36} \\
  |\left<\xi_3, \psi\right>|^2
  &=& \left(\frac{1}{6}(2 + 2 - 1 + 0)\right)^2
  = \left(\frac{1}{2}\right)^2 = \frac{1}{4} \\
  |\left<\xi_4, \psi\right>|^2
  &=& \left(\frac{1}{6}(2 + 2 + 1 - 0)\right)^2
  = \left(\frac{5}{6}\right)^2 = \frac{25}{36}
\end{eqnarray*}
となる. 公理2より, 数値1, 2, 3, 4が検出される確率はそれぞれ
$\frac{1}{36}, \frac{1}{36}, \frac{1}{4}, \frac{25}{36}$
である.

\section{}
$m, n \in \mathbb{N}$とし, 集合の元どうしで内積を計算する.
$x \in [0, 2\pi]$のとき$\cos{nx}, \sin{nx} \in [0, 1]$なので, 
\begin{eqnarray*}
  \overline{\cos{nx}} = \cos{nx} \\
  \overline{\sin{nx}} = \sin{nx} \\
\end{eqnarray*}
が成立する. まず, 元が同じ場合の内積を計算する.
\begin{itemize}
  \item cos
    \begin{eqnarray*}
      && \left<\frac{1}{\sqrt{\pi}}\cos{nx}, \frac{1}{\sqrt{\pi}}\cos{nx}\right>
      = \int_{0}^{2\pi}{\overline{\frac{1}{\sqrt{\pi}}\cos{nx}}\frac{1}{\sqrt{\pi}}\cos{nx}}dx \\
      &=& \frac{1}{\pi}\int_{0}^{2\pi}{\cos^{2}{nx}}dx
      = \frac{1}{2\pi}\int_{0}^{2\pi}{\left(1 + \cos{2nx}\right)}dx \\
      &=& \frac{1}{2\pi}\left[x + \frac{1}{2}\sin{2nx}\right]_{0}^{2\pi}
      = \frac{1}{2\pi}\left(2\pi - 0\right) = 1
    \end{eqnarray*}
  \item sin
    \begin{eqnarray*}
      && \left<\frac{1}{\sqrt{\pi}}\sin{nx}, \frac{1}{\sqrt{\pi}}\sin{nx}\right>
      = \int_{0}^{2\pi}{\overline{\frac{1}{\sqrt{\pi}}\sin{nx}}\frac{1}{\sqrt{\pi}}\sin{nx}}dx \\
      &=& \frac{1}{\pi}\int_{0}^{2\pi}{\sin^{2}{nx}}dx
      = \frac{1}{2\pi}\int_{0}^{2\pi}{\left(1 - \cos{2nx}\right)}dx \\
      &=& \frac{1}{2\pi}\left[x - \frac{1}{2}\sin{2nx}\right]_{0}^{2\pi}
      = \frac{1}{2\pi}\left(2\pi - 0\right) = 1
    \end{eqnarray*}
\end{itemize}
続いて, 元が異なる場合の内積を計算する.
\begin{itemize}
  \item cos ($m \neq n$)
    \begin{eqnarray*}
      && \left<\frac{1}{\sqrt{\pi}}\cos{mx}, \frac{1}{\sqrt{\pi}}\cos{nx}\right>
      = \int_{0}^{2\pi}{\overline{\frac{1}{\sqrt{\pi}}\cos{mx}}\frac{1}{\sqrt{\pi}}\cos{nx}}dx \\
      &=& \frac{1}{\pi}\int_{0}^{2\pi}{\cos{mx}\cos{nx}}dx
      = \frac{1}{2\pi}\int_{0}^{2\pi}{\left(\cos{(m + n)x} + \cos{(m - n)x}\right)}dx \\
      &=& \frac{1}{2\pi}\left[\frac{1}{m + n}\sin{(m + n)x} + \frac{1}{m - n}\sin{(m - n)x}\right]_{0}^{2\pi}
      = \frac{1}{2\pi}\left(0 - 0\right) = 0
    \end{eqnarray*}
  \item sin ($m \neq n$)
    \begin{eqnarray*}
      && \left<\frac{1}{\sqrt{\pi}}\sin{mx}, \frac{1}{\sqrt{\pi}}\sin{nx}\right>
      = \int_{0}^{2\pi}{\overline{\frac{1}{\sqrt{\pi}}\sin{mx}}\frac{1}{\sqrt{\pi}}\sin{nx}}dx \\
      &=& \frac{1}{\pi}\int_{0}^{2\pi}{\sin{mx}\sin{nx}}dx
      = \frac{1}{2\pi}\int_{0}^{2\pi}{\left(\cos{(m - n)x} - \cos{(m + n)x}\right)}dx \\
      &=& \frac{1}{2\pi}\left[\frac{1}{m - n}\sin{(m - n)x} - \frac{1}{m + n}\sin{(m + n)x}\right]_{0}^{2\pi}
      = \frac{1}{2\pi}\left(0 - 0\right) = 0
    \end{eqnarray*}
  \item cosとsin
    \begin{eqnarray*}
      && \left<\frac{1}{\sqrt{\pi}}\cos{mx}, \frac{1}{\sqrt{\pi}}\sin{nx}\right>
      = \int_{0}^{2\pi}{\overline{\frac{1}{\sqrt{\pi}}\cos{mx}}\frac{1}{\sqrt{\pi}}\sin{nx}}dx \\
      &=& \frac{1}{\pi}\int_{0}^{2\pi}{\cos{mx}\sin{nx}}dx
      = \frac{1}{2\pi}\int_{0}^{2\pi}{\left(\sin{(m + n)x} + \sin{(m - n)x}\right)}dx \\
      &=& \frac{1}{2\pi}\left[-\frac{1}{m + n}\cos{(m + n)x} - \frac{1}{m - n}\cos{(m - n)x}\right]_{0}^{2\pi} \\
      &=& \frac{1}{2\pi}\left(\left(-\frac{1}{m + n} -\frac{1}{m - n}\right)-\left(-\frac{1}{m + n} -\frac{1}{m - n}\right)\right) = 0
    \end{eqnarray*}
  \item sinとcos
    \begin{eqnarray*}
      && \left<\frac{1}{\sqrt{\pi}}\sin{mx}, \frac{1}{\sqrt{\pi}}\cos{nx}\right>
      = \int_{0}^{2\pi}{\overline{\frac{1}{\sqrt{\pi}}\sin{mx}}\frac{1}{\sqrt{\pi}}\cos{nx}}dx \\
      &=& \frac{1}{\pi}\int_{0}^{2\pi}{\cos{nx}\sin{mx}}dx
      = 0
    \end{eqnarray*}
\end{itemize}
以上より, 同じ元どうしの内積が1になり, 
異なる元どうしの内積が0になるので, 
\[
  \Set{\frac{1}{\sqrt{\pi}}\cos{nx} | n \in \mathbb{N}}
  \cup \Set{\frac{1}{\sqrt{\pi}}\sin{nx} | n \in \mathbb{N}}
\]
は正規直交系である.

\section{}
\begin{eqnarray*}
  \left<\xi, \zeta\right> &=& \left<
    \frac{1}{\sqrt{n}}\sum_{i=1}^{n}e_i \otimes e_i,
    \frac{1}{n}\sum_{j=1}^{n}\sum_{k=1}^{n}e_j \otimes e_k
  \right> \\
  &=& \frac{1}{n\sqrt{n}}\sum_{i=1}^n\sum_{j=1}^n\sum_{k=1}^n
  \left<e_i, e_j\right> \cdot \left<e_i, e_k\right>
\end{eqnarray*}

$\Set{e_i}_{i=1}^n$は$\mathbb{C}^n$の正規直交基底なので, 
$\left<e_i, e_j\right> \cdot \left<e_i, e_k\right>$は
$e_i = e_j = e_k$のときのみ1になり, それ以外は0となる.
よって

\begin{eqnarray*}
  \left<\xi, \zeta\right>
  &=& \frac{1}{n\sqrt{n}}\sum_{i=1}^n\sum_{j=1}^n\sum_{k=1}^n
    \left<e_i, e_j\right> \cdot \left<e_i, e_k\right> \\
  &=& \frac{1}{n\sqrt{n}}\sum_{i=1}^{n}{1}
    = \frac{1}{n\sqrt{n}} \cdot n = \frac{1}{\sqrt{n}}
\end{eqnarray*}

\section{}
\begin{eqnarray*}
  Ty &=& \left(\sum_{i=1}^n\Ket{x_i}\Bra{x_i}\right)y
    = \sum_{i=1}^n\Ket{x_i}\Bra{x_i}y \\
  &=& \sum_{i=1}^n\left<x_i, y\right>x_i
    = \sum_{i=1}^n\left<x_i, \hspace{2pt}\sum_{j=1}^n\alpha_j x_j\right>x_i \\
  &=& \sum_{i=1}^n\sum_{j=1}^n\alpha_j\left<x_i, x_j\right>x_i
    = \sum_{i=1}^n\alpha_ix_i = y
\end{eqnarray*}

\section{}
まず, 
\begin{eqnarray*}
  \frac{d}{dx}^*\cos{x} = a_1\cos{x} + b_1\sin{x} \\
  \frac{d}{dx}^*\sin{x} = a_2\cos{x} + b_2\sin{x}
\end{eqnarray*}
と置く. cosとsinの内積を計算すると, それぞれ
\begin{eqnarray*}
  \left<\cos{x}, \cos{x}\right> &=& \int_0^{2\pi}\cos^2{x}dx = \int_0^{2\pi}\frac{1+\cos{2x}}{2}dx \\
  &=& \frac{1}{2}\left[x + \frac{1}{2}\sin{2x}\right]_0^{2\pi}
  = \frac{1}{2}\left(2\pi + 0\right) = \pi \\
  \\\\
  \left<\sin{x}, \sin{x}\right> &=& \int_0^{2\pi}\sin^2{x}dx = \int_0^{2\pi}\frac{1-\cos{2x}}{2}dx \\
  &=& \frac{1}{2}\left[x - \frac{1}{2}\sin{2x}\right]_0^{2\pi}
  = \frac{1}{2}\left(2\pi - 0\right) = \pi \\
  \\\\
  \left<\cos{x}, \sin{x}\right> &=& \int_0^{2\pi}\cos{x}\sin{x}dx
    = \int_0^{2\pi}\frac{1}{2}\sin{2x}dx \\
  &=& \frac{1}{2}\left[-\frac{1}{2}\cos{2x}\right]_0^{2\pi}
  = \frac{1}{2}\left(-\frac{1}{2} + \frac{1}{2}\right) = 0 \\
  \\\\
  \left<\sin{x}, \cos{x}\right> &=& \int_0^{2\pi}\sin{x}\cos{x}dx = 0
\end{eqnarray*}
となる. 

$a_1, b_1$を求めるために 
$\left<\frac{d}{dx}\cos{x}, \cos{x}\right>$,
$\left<\cos{x}, \frac{d}{dx}^*\cos{x}\right>$,
$\left<\frac{d}{dx}\sin{x}, \cos{x}\right>$,
$\left<\sin{x}, \frac{d}{dx}^*\cos{x}\right>$
を計算する.
\begin{eqnarray}
  \left<\frac{d}{dx}\cos{x}, \cos{x}\right>
  = \left<-\sin{x}, \cos{x}\right>
  = -\left<\sin{x}, \cos{x}\right>
  = 0 \label{dcc}
\end{eqnarray}
\begin{equation}
  \begin{split}
    \left<\cos{x}, \frac{d}{dx}^*\cos{x}\right>
    = \left<\cos{x}, a_1\cos{x} + b_1\sin{x}\right> \\
    = a_1\left<\cos{x}, \cos{x}\right> + b_1\left<\cos{x}, \sin{x}\right>
    = \pi a_1
  \end{split}
  \label{cadc}
\end{equation}
随伴作用素の定義と\eqref{dcc}, \eqref{cadc}より, $a_1=0$である.
\begin{eqnarray}
  \left<\frac{d}{dx}\sin{x}, \cos{x}\right>
  = \left<\cos{x}, \cos{x}\right>
  = \pi \label{dsc}
\end{eqnarray}
\begin{equation}
  \begin{split}
    \left<\sin{x}, \frac{d}{dx}^*\cos{x}\right>
    = \left<\sin{x}, a_1\cos{x} + b_1\sin{x}\right> \\
    = a_1\left<\sin{x}, \cos{x}\right> + b_1\left<\sin{x}, \sin{x}\right>
    = \pi b_1
  \end{split}
  \label{sadc}
\end{equation}
随伴作用素の定義と\eqref{dsc}, \eqref{sadc}より, $b_1=1$である.
以上より, 
\[
  \frac{d}{dx}^*\cos{x} = \sin{x}
\]
が示された.

$a_2, b_2$を求めるために 
$\left<\frac{d}{dx}\sin{x}, \sin{x}\right>$,
$\left<\sin{x}, \frac{d}{dx}^*\sin{x}\right>$,
$\left<\frac{d}{dx}\cos{x}, \sin{x}\right>$,
$\left<\cos{x}, \frac{d}{dx}^*\sin{x}\right>$
を計算する.
\begin{eqnarray}
  \left<\frac{d}{dx}\sin{x}, \sin{x}\right>
  = \left<\cos{x}, \sin{x}\right>
  = 0 \label{dss}
\end{eqnarray}
\begin{equation}
  \begin{split}
    \left<\sin{x}, \frac{d}{dx}^*\sin{x}\right>
    = \left<\sin{x}, a_2\cos{x} + b_2\sin{x}\right> \\
    = a_2\left<\sin{x}, \cos{x}\right> + b_2\left<\sin{x}, \sin{x}\right>
    = \pi b_2
  \end{split}
  \label{sads}
\end{equation}
随伴作用素の定義と\eqref{dss}, \eqref{sads}より, $b_2=0$である.
\begin{eqnarray}
  \left<\frac{d}{dx}\cos{x}, \sin{x}\right>
  = \left<-\sin{x}, \sin{x}\right>
  = -\left<\sin{x}, \sin{x}\right>
  = -\pi \label{dcs}
\end{eqnarray}
\begin{equation}
  \begin{split}
    \left<\cos{x}, \frac{d}{dx}^*\sin{x}\right>
    = \left<\cos{x}, a_2\cos{x} + b_2\sin{x}\right> \\
    = a_2\left<\cos{x}, \cos{x}\right> + b_2\left<\cos{x}, \sin{x}\right>
    = \pi a_2
  \end{split}
  \label{cads}
\end{equation}
随伴作用素の定義と\eqref{dcs}, \eqref{cads}より, $a_2=-1$である.
以上より, 
\[
  \frac{d}{dx}^*\cos{x} = \sin{x}
\]
が示された.

\end{document}
